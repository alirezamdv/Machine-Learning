\documentclass{dfki_ml}
% \exsheetheader{exercise sheet number}{hand in date}{group number}{estimated time in hours}
%               {group member 1}{group member 2}{group member 3}{group member 4}

\exsheetheader{1}{01.01.2018}{42}{2}{Max Musterman}{}{}{}

\begin{document}

\begin{problem}{Instructions}
This is an example usage of the template. In this problem
we will use an inline python code. One very useful library
is \textbf{numpy} and can be imported as shown: \pythoninline{import numpy as np}.
\end{problem}


\begin{problem}{Larger Python sections}
If we want to refer to a chunk of python code we can use the python
environment.
\begin{python}
import numpy as np

def random_number():
    return 4 # Chosen by fair dice roll :)
    
print(random_number())
\end{python}
\end{problem}

\begin{problem}{Reduce redundancy}
And as we like to make life easier for the people who have to
correct our solutions we always should append the original python
source files (as they are handed in as well) at the end of the 
document. This can be done by simply referring to a python file
with the command \textbf{pythonexternal}:

\pythonexternal{example.py}

\end{problem}

\end{document}
